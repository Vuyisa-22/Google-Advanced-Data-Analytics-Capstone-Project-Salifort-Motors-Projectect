\documentclass[11pt]{article}

    \usepackage[breakable]{tcolorbox}
    \usepackage{parskip} % Stop auto-indenting (to mimic markdown behaviour)
    
    \usepackage{iftex}
    \ifPDFTeX
    	\usepackage[T1]{fontenc}
    	\usepackage{mathpazo}
    \else
    	\usepackage{fontspec}
    \fi

    % Basic figure setup, for now with no caption control since it's done
    % automatically by Pandoc (which extracts ![](path) syntax from Markdown).
    \usepackage{graphicx}
    % Maintain compatibility with old templates. Remove in nbconvert 6.0
    \let\Oldincludegraphics\includegraphics
    % Ensure that by default, figures have no caption (until we provide a
    % proper Figure object with a Caption API and a way to capture that
    % in the conversion process - todo).
    \usepackage{caption}
    \DeclareCaptionFormat{nocaption}{}
    \captionsetup{format=nocaption,aboveskip=0pt,belowskip=0pt}

    \usepackage[Export]{adjustbox} % Used to constrain images to a maximum size
    \adjustboxset{max size={0.9\linewidth}{0.9\paperheight}}
    \usepackage{float}
    \floatplacement{figure}{H} % forces figures to be placed at the correct location
    \usepackage{xcolor} % Allow colors to be defined
    \usepackage{enumerate} % Needed for markdown enumerations to work
    \usepackage{geometry} % Used to adjust the document margins
    \usepackage{amsmath} % Equations
    \usepackage{amssymb} % Equations
    \usepackage{textcomp} % defines textquotesingle
    % Hack from http://tex.stackexchange.com/a/47451/13684:
    \AtBeginDocument{%
        \def\PYZsq{\textquotesingle}% Upright quotes in Pygmentized code
    }
    \usepackage{upquote} % Upright quotes for verbatim code
    \usepackage{eurosym} % defines \euro
    \usepackage[mathletters]{ucs} % Extended unicode (utf-8) support
    \usepackage{fancyvrb} % verbatim replacement that allows latex
    \usepackage{grffile} % extends the file name processing of package graphics 
                         % to support a larger range
    \makeatletter % fix for grffile with XeLaTeX
    \def\Gread@@xetex#1{%
      \IfFileExists{"\Gin@base".bb}%
      {\Gread@eps{\Gin@base.bb}}%
      {\Gread@@xetex@aux#1}%
    }
    \makeatother

    % The hyperref package gives us a pdf with properly built
    % internal navigation ('pdf bookmarks' for the table of contents,
    % internal cross-reference links, web links for URLs, etc.)
    \usepackage{hyperref}
    % The default LaTeX title has an obnoxious amount of whitespace. By default,
    % titling removes some of it. It also provides customization options.
    \usepackage{titling}
    \usepackage{longtable} % longtable support required by pandoc >1.10
    \usepackage{booktabs}  % table support for pandoc > 1.12.2
    \usepackage[inline]{enumitem} % IRkernel/repr support (it uses the enumerate* environment)
    \usepackage[normalem]{ulem} % ulem is needed to support strikethroughs (\sout)
                                % normalem makes italics be italics, not underlines
    \usepackage{mathrsfs}
    

    
    % Colors for the hyperref package
    \definecolor{urlcolor}{rgb}{0,.145,.698}
    \definecolor{linkcolor}{rgb}{.71,0.21,0.01}
    \definecolor{citecolor}{rgb}{.12,.54,.11}

    % ANSI colors
    \definecolor{ansi-black}{HTML}{3E424D}
    \definecolor{ansi-black-intense}{HTML}{282C36}
    \definecolor{ansi-red}{HTML}{E75C58}
    \definecolor{ansi-red-intense}{HTML}{B22B31}
    \definecolor{ansi-green}{HTML}{00A250}
    \definecolor{ansi-green-intense}{HTML}{007427}
    \definecolor{ansi-yellow}{HTML}{DDB62B}
    \definecolor{ansi-yellow-intense}{HTML}{B27D12}
    \definecolor{ansi-blue}{HTML}{208FFB}
    \definecolor{ansi-blue-intense}{HTML}{0065CA}
    \definecolor{ansi-magenta}{HTML}{D160C4}
    \definecolor{ansi-magenta-intense}{HTML}{A03196}
    \definecolor{ansi-cyan}{HTML}{60C6C8}
    \definecolor{ansi-cyan-intense}{HTML}{258F8F}
    \definecolor{ansi-white}{HTML}{C5C1B4}
    \definecolor{ansi-white-intense}{HTML}{A1A6B2}
    \definecolor{ansi-default-inverse-fg}{HTML}{FFFFFF}
    \definecolor{ansi-default-inverse-bg}{HTML}{000000}

    % commands and environments needed by pandoc snippets
    % extracted from the output of `pandoc -s`
    \providecommand{\tightlist}{%
      \setlength{\itemsep}{0pt}\setlength{\parskip}{0pt}}
    \DefineVerbatimEnvironment{Highlighting}{Verbatim}{commandchars=\\\{\}}
    % Add ',fontsize=\small' for more characters per line
    \newenvironment{Shaded}{}{}
    \newcommand{\KeywordTok}[1]{\textcolor[rgb]{0.00,0.44,0.13}{\textbf{{#1}}}}
    \newcommand{\DataTypeTok}[1]{\textcolor[rgb]{0.56,0.13,0.00}{{#1}}}
    \newcommand{\DecValTok}[1]{\textcolor[rgb]{0.25,0.63,0.44}{{#1}}}
    \newcommand{\BaseNTok}[1]{\textcolor[rgb]{0.25,0.63,0.44}{{#1}}}
    \newcommand{\FloatTok}[1]{\textcolor[rgb]{0.25,0.63,0.44}{{#1}}}
    \newcommand{\CharTok}[1]{\textcolor[rgb]{0.25,0.44,0.63}{{#1}}}
    \newcommand{\StringTok}[1]{\textcolor[rgb]{0.25,0.44,0.63}{{#1}}}
    \newcommand{\CommentTok}[1]{\textcolor[rgb]{0.38,0.63,0.69}{\textit{{#1}}}}
    \newcommand{\OtherTok}[1]{\textcolor[rgb]{0.00,0.44,0.13}{{#1}}}
    \newcommand{\AlertTok}[1]{\textcolor[rgb]{1.00,0.00,0.00}{\textbf{{#1}}}}
    \newcommand{\FunctionTok}[1]{\textcolor[rgb]{0.02,0.16,0.49}{{#1}}}
    \newcommand{\RegionMarkerTok}[1]{{#1}}
    \newcommand{\ErrorTok}[1]{\textcolor[rgb]{1.00,0.00,0.00}{\textbf{{#1}}}}
    \newcommand{\NormalTok}[1]{{#1}}
    
    % Additional commands for more recent versions of Pandoc
    \newcommand{\ConstantTok}[1]{\textcolor[rgb]{0.53,0.00,0.00}{{#1}}}
    \newcommand{\SpecialCharTok}[1]{\textcolor[rgb]{0.25,0.44,0.63}{{#1}}}
    \newcommand{\VerbatimStringTok}[1]{\textcolor[rgb]{0.25,0.44,0.63}{{#1}}}
    \newcommand{\SpecialStringTok}[1]{\textcolor[rgb]{0.73,0.40,0.53}{{#1}}}
    \newcommand{\ImportTok}[1]{{#1}}
    \newcommand{\DocumentationTok}[1]{\textcolor[rgb]{0.73,0.13,0.13}{\textit{{#1}}}}
    \newcommand{\AnnotationTok}[1]{\textcolor[rgb]{0.38,0.63,0.69}{\textbf{\textit{{#1}}}}}
    \newcommand{\CommentVarTok}[1]{\textcolor[rgb]{0.38,0.63,0.69}{\textbf{\textit{{#1}}}}}
    \newcommand{\VariableTok}[1]{\textcolor[rgb]{0.10,0.09,0.49}{{#1}}}
    \newcommand{\ControlFlowTok}[1]{\textcolor[rgb]{0.00,0.44,0.13}{\textbf{{#1}}}}
    \newcommand{\OperatorTok}[1]{\textcolor[rgb]{0.40,0.40,0.40}{{#1}}}
    \newcommand{\BuiltInTok}[1]{{#1}}
    \newcommand{\ExtensionTok}[1]{{#1}}
    \newcommand{\PreprocessorTok}[1]{\textcolor[rgb]{0.74,0.48,0.00}{{#1}}}
    \newcommand{\AttributeTok}[1]{\textcolor[rgb]{0.49,0.56,0.16}{{#1}}}
    \newcommand{\InformationTok}[1]{\textcolor[rgb]{0.38,0.63,0.69}{\textbf{\textit{{#1}}}}}
    \newcommand{\WarningTok}[1]{\textcolor[rgb]{0.38,0.63,0.69}{\textbf{\textit{{#1}}}}}
    
    
    % Define a nice break command that doesn't care if a line doesn't already
    % exist.
    \def\br{\hspace*{\fill} \\* }
    % Math Jax compatibility definitions
    \def\gt{>}
    \def\lt{<}
    \let\Oldtex\TeX
    \let\Oldlatex\LaTeX
    \renewcommand{\TeX}{\textrm{\Oldtex}}
    \renewcommand{\LaTeX}{\textrm{\Oldlatex}}
    % Document parameters
    % Document title
    \title{Activity\_ Course 7 Salifort Motors project lab-Copy1}
    
    
    
    
    
% Pygments definitions
\makeatletter
\def\PY@reset{\let\PY@it=\relax \let\PY@bf=\relax%
    \let\PY@ul=\relax \let\PY@tc=\relax%
    \let\PY@bc=\relax \let\PY@ff=\relax}
\def\PY@tok#1{\csname PY@tok@#1\endcsname}
\def\PY@toks#1+{\ifx\relax#1\empty\else%
    \PY@tok{#1}\expandafter\PY@toks\fi}
\def\PY@do#1{\PY@bc{\PY@tc{\PY@ul{%
    \PY@it{\PY@bf{\PY@ff{#1}}}}}}}
\def\PY#1#2{\PY@reset\PY@toks#1+\relax+\PY@do{#2}}

\expandafter\def\csname PY@tok@w\endcsname{\def\PY@tc##1{\textcolor[rgb]{0.73,0.73,0.73}{##1}}}
\expandafter\def\csname PY@tok@c\endcsname{\let\PY@it=\textit\def\PY@tc##1{\textcolor[rgb]{0.25,0.50,0.50}{##1}}}
\expandafter\def\csname PY@tok@cp\endcsname{\def\PY@tc##1{\textcolor[rgb]{0.74,0.48,0.00}{##1}}}
\expandafter\def\csname PY@tok@k\endcsname{\let\PY@bf=\textbf\def\PY@tc##1{\textcolor[rgb]{0.00,0.50,0.00}{##1}}}
\expandafter\def\csname PY@tok@kp\endcsname{\def\PY@tc##1{\textcolor[rgb]{0.00,0.50,0.00}{##1}}}
\expandafter\def\csname PY@tok@kt\endcsname{\def\PY@tc##1{\textcolor[rgb]{0.69,0.00,0.25}{##1}}}
\expandafter\def\csname PY@tok@o\endcsname{\def\PY@tc##1{\textcolor[rgb]{0.40,0.40,0.40}{##1}}}
\expandafter\def\csname PY@tok@ow\endcsname{\let\PY@bf=\textbf\def\PY@tc##1{\textcolor[rgb]{0.67,0.13,1.00}{##1}}}
\expandafter\def\csname PY@tok@nb\endcsname{\def\PY@tc##1{\textcolor[rgb]{0.00,0.50,0.00}{##1}}}
\expandafter\def\csname PY@tok@nf\endcsname{\def\PY@tc##1{\textcolor[rgb]{0.00,0.00,1.00}{##1}}}
\expandafter\def\csname PY@tok@nc\endcsname{\let\PY@bf=\textbf\def\PY@tc##1{\textcolor[rgb]{0.00,0.00,1.00}{##1}}}
\expandafter\def\csname PY@tok@nn\endcsname{\let\PY@bf=\textbf\def\PY@tc##1{\textcolor[rgb]{0.00,0.00,1.00}{##1}}}
\expandafter\def\csname PY@tok@ne\endcsname{\let\PY@bf=\textbf\def\PY@tc##1{\textcolor[rgb]{0.82,0.25,0.23}{##1}}}
\expandafter\def\csname PY@tok@nv\endcsname{\def\PY@tc##1{\textcolor[rgb]{0.10,0.09,0.49}{##1}}}
\expandafter\def\csname PY@tok@no\endcsname{\def\PY@tc##1{\textcolor[rgb]{0.53,0.00,0.00}{##1}}}
\expandafter\def\csname PY@tok@nl\endcsname{\def\PY@tc##1{\textcolor[rgb]{0.63,0.63,0.00}{##1}}}
\expandafter\def\csname PY@tok@ni\endcsname{\let\PY@bf=\textbf\def\PY@tc##1{\textcolor[rgb]{0.60,0.60,0.60}{##1}}}
\expandafter\def\csname PY@tok@na\endcsname{\def\PY@tc##1{\textcolor[rgb]{0.49,0.56,0.16}{##1}}}
\expandafter\def\csname PY@tok@nt\endcsname{\let\PY@bf=\textbf\def\PY@tc##1{\textcolor[rgb]{0.00,0.50,0.00}{##1}}}
\expandafter\def\csname PY@tok@nd\endcsname{\def\PY@tc##1{\textcolor[rgb]{0.67,0.13,1.00}{##1}}}
\expandafter\def\csname PY@tok@s\endcsname{\def\PY@tc##1{\textcolor[rgb]{0.73,0.13,0.13}{##1}}}
\expandafter\def\csname PY@tok@sd\endcsname{\let\PY@it=\textit\def\PY@tc##1{\textcolor[rgb]{0.73,0.13,0.13}{##1}}}
\expandafter\def\csname PY@tok@si\endcsname{\let\PY@bf=\textbf\def\PY@tc##1{\textcolor[rgb]{0.73,0.40,0.53}{##1}}}
\expandafter\def\csname PY@tok@se\endcsname{\let\PY@bf=\textbf\def\PY@tc##1{\textcolor[rgb]{0.73,0.40,0.13}{##1}}}
\expandafter\def\csname PY@tok@sr\endcsname{\def\PY@tc##1{\textcolor[rgb]{0.73,0.40,0.53}{##1}}}
\expandafter\def\csname PY@tok@ss\endcsname{\def\PY@tc##1{\textcolor[rgb]{0.10,0.09,0.49}{##1}}}
\expandafter\def\csname PY@tok@sx\endcsname{\def\PY@tc##1{\textcolor[rgb]{0.00,0.50,0.00}{##1}}}
\expandafter\def\csname PY@tok@m\endcsname{\def\PY@tc##1{\textcolor[rgb]{0.40,0.40,0.40}{##1}}}
\expandafter\def\csname PY@tok@gh\endcsname{\let\PY@bf=\textbf\def\PY@tc##1{\textcolor[rgb]{0.00,0.00,0.50}{##1}}}
\expandafter\def\csname PY@tok@gu\endcsname{\let\PY@bf=\textbf\def\PY@tc##1{\textcolor[rgb]{0.50,0.00,0.50}{##1}}}
\expandafter\def\csname PY@tok@gd\endcsname{\def\PY@tc##1{\textcolor[rgb]{0.63,0.00,0.00}{##1}}}
\expandafter\def\csname PY@tok@gi\endcsname{\def\PY@tc##1{\textcolor[rgb]{0.00,0.63,0.00}{##1}}}
\expandafter\def\csname PY@tok@gr\endcsname{\def\PY@tc##1{\textcolor[rgb]{1.00,0.00,0.00}{##1}}}
\expandafter\def\csname PY@tok@ge\endcsname{\let\PY@it=\textit}
\expandafter\def\csname PY@tok@gs\endcsname{\let\PY@bf=\textbf}
\expandafter\def\csname PY@tok@gp\endcsname{\let\PY@bf=\textbf\def\PY@tc##1{\textcolor[rgb]{0.00,0.00,0.50}{##1}}}
\expandafter\def\csname PY@tok@go\endcsname{\def\PY@tc##1{\textcolor[rgb]{0.53,0.53,0.53}{##1}}}
\expandafter\def\csname PY@tok@gt\endcsname{\def\PY@tc##1{\textcolor[rgb]{0.00,0.27,0.87}{##1}}}
\expandafter\def\csname PY@tok@err\endcsname{\def\PY@bc##1{\setlength{\fboxsep}{0pt}\fcolorbox[rgb]{1.00,0.00,0.00}{1,1,1}{\strut ##1}}}
\expandafter\def\csname PY@tok@kc\endcsname{\let\PY@bf=\textbf\def\PY@tc##1{\textcolor[rgb]{0.00,0.50,0.00}{##1}}}
\expandafter\def\csname PY@tok@kd\endcsname{\let\PY@bf=\textbf\def\PY@tc##1{\textcolor[rgb]{0.00,0.50,0.00}{##1}}}
\expandafter\def\csname PY@tok@kn\endcsname{\let\PY@bf=\textbf\def\PY@tc##1{\textcolor[rgb]{0.00,0.50,0.00}{##1}}}
\expandafter\def\csname PY@tok@kr\endcsname{\let\PY@bf=\textbf\def\PY@tc##1{\textcolor[rgb]{0.00,0.50,0.00}{##1}}}
\expandafter\def\csname PY@tok@bp\endcsname{\def\PY@tc##1{\textcolor[rgb]{0.00,0.50,0.00}{##1}}}
\expandafter\def\csname PY@tok@fm\endcsname{\def\PY@tc##1{\textcolor[rgb]{0.00,0.00,1.00}{##1}}}
\expandafter\def\csname PY@tok@vc\endcsname{\def\PY@tc##1{\textcolor[rgb]{0.10,0.09,0.49}{##1}}}
\expandafter\def\csname PY@tok@vg\endcsname{\def\PY@tc##1{\textcolor[rgb]{0.10,0.09,0.49}{##1}}}
\expandafter\def\csname PY@tok@vi\endcsname{\def\PY@tc##1{\textcolor[rgb]{0.10,0.09,0.49}{##1}}}
\expandafter\def\csname PY@tok@vm\endcsname{\def\PY@tc##1{\textcolor[rgb]{0.10,0.09,0.49}{##1}}}
\expandafter\def\csname PY@tok@sa\endcsname{\def\PY@tc##1{\textcolor[rgb]{0.73,0.13,0.13}{##1}}}
\expandafter\def\csname PY@tok@sb\endcsname{\def\PY@tc##1{\textcolor[rgb]{0.73,0.13,0.13}{##1}}}
\expandafter\def\csname PY@tok@sc\endcsname{\def\PY@tc##1{\textcolor[rgb]{0.73,0.13,0.13}{##1}}}
\expandafter\def\csname PY@tok@dl\endcsname{\def\PY@tc##1{\textcolor[rgb]{0.73,0.13,0.13}{##1}}}
\expandafter\def\csname PY@tok@s2\endcsname{\def\PY@tc##1{\textcolor[rgb]{0.73,0.13,0.13}{##1}}}
\expandafter\def\csname PY@tok@sh\endcsname{\def\PY@tc##1{\textcolor[rgb]{0.73,0.13,0.13}{##1}}}
\expandafter\def\csname PY@tok@s1\endcsname{\def\PY@tc##1{\textcolor[rgb]{0.73,0.13,0.13}{##1}}}
\expandafter\def\csname PY@tok@mb\endcsname{\def\PY@tc##1{\textcolor[rgb]{0.40,0.40,0.40}{##1}}}
\expandafter\def\csname PY@tok@mf\endcsname{\def\PY@tc##1{\textcolor[rgb]{0.40,0.40,0.40}{##1}}}
\expandafter\def\csname PY@tok@mh\endcsname{\def\PY@tc##1{\textcolor[rgb]{0.40,0.40,0.40}{##1}}}
\expandafter\def\csname PY@tok@mi\endcsname{\def\PY@tc##1{\textcolor[rgb]{0.40,0.40,0.40}{##1}}}
\expandafter\def\csname PY@tok@il\endcsname{\def\PY@tc##1{\textcolor[rgb]{0.40,0.40,0.40}{##1}}}
\expandafter\def\csname PY@tok@mo\endcsname{\def\PY@tc##1{\textcolor[rgb]{0.40,0.40,0.40}{##1}}}
\expandafter\def\csname PY@tok@ch\endcsname{\let\PY@it=\textit\def\PY@tc##1{\textcolor[rgb]{0.25,0.50,0.50}{##1}}}
\expandafter\def\csname PY@tok@cm\endcsname{\let\PY@it=\textit\def\PY@tc##1{\textcolor[rgb]{0.25,0.50,0.50}{##1}}}
\expandafter\def\csname PY@tok@cpf\endcsname{\let\PY@it=\textit\def\PY@tc##1{\textcolor[rgb]{0.25,0.50,0.50}{##1}}}
\expandafter\def\csname PY@tok@c1\endcsname{\let\PY@it=\textit\def\PY@tc##1{\textcolor[rgb]{0.25,0.50,0.50}{##1}}}
\expandafter\def\csname PY@tok@cs\endcsname{\let\PY@it=\textit\def\PY@tc##1{\textcolor[rgb]{0.25,0.50,0.50}{##1}}}

\def\PYZbs{\char`\\}
\def\PYZus{\char`\_}
\def\PYZob{\char`\{}
\def\PYZcb{\char`\}}
\def\PYZca{\char`\^}
\def\PYZam{\char`\&}
\def\PYZlt{\char`\<}
\def\PYZgt{\char`\>}
\def\PYZsh{\char`\#}
\def\PYZpc{\char`\%}
\def\PYZdl{\char`\$}
\def\PYZhy{\char`\-}
\def\PYZsq{\char`\'}
\def\PYZdq{\char`\"}
\def\PYZti{\char`\~}
% for compatibility with earlier versions
\def\PYZat{@}
\def\PYZlb{[}
\def\PYZrb{]}
\makeatother


    % For linebreaks inside Verbatim environment from package fancyvrb. 
    \makeatletter
        \newbox\Wrappedcontinuationbox 
        \newbox\Wrappedvisiblespacebox 
        \newcommand*\Wrappedvisiblespace {\textcolor{red}{\textvisiblespace}} 
        \newcommand*\Wrappedcontinuationsymbol {\textcolor{red}{\llap{\tiny$\m@th\hookrightarrow$}}} 
        \newcommand*\Wrappedcontinuationindent {3ex } 
        \newcommand*\Wrappedafterbreak {\kern\Wrappedcontinuationindent\copy\Wrappedcontinuationbox} 
        % Take advantage of the already applied Pygments mark-up to insert 
        % potential linebreaks for TeX processing. 
        %        {, <, #, %, $, ' and ": go to next line. 
        %        _, }, ^, &, >, - and ~: stay at end of broken line. 
        % Use of \textquotesingle for straight quote. 
        \newcommand*\Wrappedbreaksatspecials {% 
            \def\PYGZus{\discretionary{\char`\_}{\Wrappedafterbreak}{\char`\_}}% 
            \def\PYGZob{\discretionary{}{\Wrappedafterbreak\char`\{}{\char`\{}}% 
            \def\PYGZcb{\discretionary{\char`\}}{\Wrappedafterbreak}{\char`\}}}% 
            \def\PYGZca{\discretionary{\char`\^}{\Wrappedafterbreak}{\char`\^}}% 
            \def\PYGZam{\discretionary{\char`\&}{\Wrappedafterbreak}{\char`\&}}% 
            \def\PYGZlt{\discretionary{}{\Wrappedafterbreak\char`\<}{\char`\<}}% 
            \def\PYGZgt{\discretionary{\char`\>}{\Wrappedafterbreak}{\char`\>}}% 
            \def\PYGZsh{\discretionary{}{\Wrappedafterbreak\char`\#}{\char`\#}}% 
            \def\PYGZpc{\discretionary{}{\Wrappedafterbreak\char`\%}{\char`\%}}% 
            \def\PYGZdl{\discretionary{}{\Wrappedafterbreak\char`\$}{\char`\$}}% 
            \def\PYGZhy{\discretionary{\char`\-}{\Wrappedafterbreak}{\char`\-}}% 
            \def\PYGZsq{\discretionary{}{\Wrappedafterbreak\textquotesingle}{\textquotesingle}}% 
            \def\PYGZdq{\discretionary{}{\Wrappedafterbreak\char`\"}{\char`\"}}% 
            \def\PYGZti{\discretionary{\char`\~}{\Wrappedafterbreak}{\char`\~}}% 
        } 
        % Some characters . , ; ? ! / are not pygmentized. 
        % This macro makes them "active" and they will insert potential linebreaks 
        \newcommand*\Wrappedbreaksatpunct {% 
            \lccode`\~`\.\lowercase{\def~}{\discretionary{\hbox{\char`\.}}{\Wrappedafterbreak}{\hbox{\char`\.}}}% 
            \lccode`\~`\,\lowercase{\def~}{\discretionary{\hbox{\char`\,}}{\Wrappedafterbreak}{\hbox{\char`\,}}}% 
            \lccode`\~`\;\lowercase{\def~}{\discretionary{\hbox{\char`\;}}{\Wrappedafterbreak}{\hbox{\char`\;}}}% 
            \lccode`\~`\:\lowercase{\def~}{\discretionary{\hbox{\char`\:}}{\Wrappedafterbreak}{\hbox{\char`\:}}}% 
            \lccode`\~`\?\lowercase{\def~}{\discretionary{\hbox{\char`\?}}{\Wrappedafterbreak}{\hbox{\char`\?}}}% 
            \lccode`\~`\!\lowercase{\def~}{\discretionary{\hbox{\char`\!}}{\Wrappedafterbreak}{\hbox{\char`\!}}}% 
            \lccode`\~`\/\lowercase{\def~}{\discretionary{\hbox{\char`\/}}{\Wrappedafterbreak}{\hbox{\char`\/}}}% 
            \catcode`\.\active
            \catcode`\,\active 
            \catcode`\;\active
            \catcode`\:\active
            \catcode`\?\active
            \catcode`\!\active
            \catcode`\/\active 
            \lccode`\~`\~ 	
        }
    \makeatother

    \let\OriginalVerbatim=\Verbatim
    \makeatletter
    \renewcommand{\Verbatim}[1][1]{%
        %\parskip\z@skip
        \sbox\Wrappedcontinuationbox {\Wrappedcontinuationsymbol}%
        \sbox\Wrappedvisiblespacebox {\FV@SetupFont\Wrappedvisiblespace}%
        \def\FancyVerbFormatLine ##1{\hsize\linewidth
            \vtop{\raggedright\hyphenpenalty\z@\exhyphenpenalty\z@
                \doublehyphendemerits\z@\finalhyphendemerits\z@
                \strut ##1\strut}%
        }%
        % If the linebreak is at a space, the latter will be displayed as visible
        % space at end of first line, and a continuation symbol starts next line.
        % Stretch/shrink are however usually zero for typewriter font.
        \def\FV@Space {%
            \nobreak\hskip\z@ plus\fontdimen3\font minus\fontdimen4\font
            \discretionary{\copy\Wrappedvisiblespacebox}{\Wrappedafterbreak}
            {\kern\fontdimen2\font}%
        }%
        
        % Allow breaks at special characters using \PYG... macros.
        \Wrappedbreaksatspecials
        % Breaks at punctuation characters . , ; ? ! and / need catcode=\active 	
        \OriginalVerbatim[#1,codes*=\Wrappedbreaksatpunct]%
    }
    \makeatother

    % Exact colors from NB
    \definecolor{incolor}{HTML}{303F9F}
    \definecolor{outcolor}{HTML}{D84315}
    \definecolor{cellborder}{HTML}{CFCFCF}
    \definecolor{cellbackground}{HTML}{F7F7F7}
    
    % prompt
    \makeatletter
    \newcommand{\boxspacing}{\kern\kvtcb@left@rule\kern\kvtcb@boxsep}
    \makeatother
    \newcommand{\prompt}[4]{
        \ttfamily\llap{{\color{#2}[#3]:\hspace{3pt}#4}}\vspace{-\baselineskip}
    }
    

    
    % Prevent overflowing lines due to hard-to-break entities
    \sloppy 
    % Setup hyperref package
    \hypersetup{
      breaklinks=true,  % so long urls are correctly broken across lines
      colorlinks=true,
      urlcolor=urlcolor,
      linkcolor=linkcolor,
      citecolor=citecolor,
      }
    % Slightly bigger margins than the latex defaults
    
    \geometry{verbose,tmargin=1in,bmargin=1in,lmargin=1in,rmargin=1in}
    
    

\begin{document}
    
    \maketitle
    
    

    
    \hypertarget{capstone-project-providing-data-driven-suggestions-for-hr}{%
\section{\texorpdfstring{\textbf{Capstone project: Providing data-driven
suggestions for
HR}}{Capstone project: Providing data-driven suggestions for HR}}\label{capstone-project-providing-data-driven-suggestions-for-hr}}

    \hypertarget{description-and-deliverables}{%
\subsection{Description and
deliverables}\label{description-and-deliverables}}

This capstone project is an opportunity for you to analyze a dataset and
build predictive models that can provide insights to the Human Resources
(HR) department of a large consulting firm.

Upon completion, you will have two artifacts that you would be able to
present to future employers. One is a brief one-page summary of this
project that you would present to external stakeholders as the data
professional in Salifort Motors. The other is a complete code notebook
provided here. Please consider your prior course work and select one way
to achieve this given project question. Either use a regression model or
machine learning model to predict whether or not an employee will leave
the company. The exemplar following this actiivty shows both approaches,
but you only need to do one.

In your deliverables, you will include the model evaluation (and
interpretation if applicable), a data visualization(s) of your choice
that is directly related to the question you ask, ethical
considerations, and the resources you used to troubleshoot and find
answers or solutions.

    \hypertarget{pace-stages}{%
\section{\texorpdfstring{\textbf{PACE
stages}}{PACE stages}}\label{pace-stages}}

    \begin{figure}
\centering
\caption{Screenshot 2022-08-04 5.47.37 PM.png}
\end{figure}

    \hypertarget{pace-plan}{%
\subsection{\texorpdfstring{\textbf{Pace:
Plan}}{Pace: Plan}}\label{pace-plan}}

Consider the questions in your PACE Strategy Document to reflect on the
Plan stage.

In this stage, consider the following:

    \hypertarget{understand-the-business-scenario-and-problem}{%
\subsubsection{Understand the business scenario and
problem}\label{understand-the-business-scenario-and-problem}}

The HR department at Salifort Motors wants to take some initiatives to
improve employee satisfaction levels at the company. They collected data
from employees, but now they don't know what to do with it. They refer
to you as a data analytics professional and ask you to provide
data-driven suggestions based on your understanding of the data. They
have the following question: what's likely to make the employee leave
the company?

Your goals in this project are to analyze the data collected by the HR
department and to build a model that predicts whether or not an employee
will leave the company.

If you can predict employees likely to quit, it might be possible to
identify factors that contribute to their leaving. Because it is
time-consuming and expensive to find, interview, and hire new employees,
increasing employee retention will be beneficial to the company.

    \hypertarget{familiarize-yourself-with-the-hr-dataset}{%
\subsubsection{Familiarize yourself with the HR
dataset}\label{familiarize-yourself-with-the-hr-dataset}}

The dataset that you'll be using in this lab contains 15,000 rows and 10
columns for the variables listed below.

\textbf{Note:} you don't need to download any data to complete this lab.
For more information about the data, refer to its source on
\href{https://www.kaggle.com/datasets/mfaisalqureshi/hr-analytics-and-job-prediction?select=HR_comma_sep.csv}{Kaggle}.

\begin{longtable}[]{@{}ll@{}}
\toprule
\begin{minipage}[b]{0.47\columnwidth}\raggedright
Variable\strut
\end{minipage} & \begin{minipage}[b]{0.47\columnwidth}\raggedright
Description\strut
\end{minipage}\tabularnewline
\midrule
\endhead
\begin{minipage}[t]{0.47\columnwidth}\raggedright
satisfaction\_level\strut
\end{minipage} & \begin{minipage}[t]{0.47\columnwidth}\raggedright
Employee-reported job satisfaction level {[}0--1{]}\strut
\end{minipage}\tabularnewline
\begin{minipage}[t]{0.47\columnwidth}\raggedright
last\_evaluation\strut
\end{minipage} & \begin{minipage}[t]{0.47\columnwidth}\raggedright
Score of employee's last performance review {[}0--1{]}\strut
\end{minipage}\tabularnewline
\begin{minipage}[t]{0.47\columnwidth}\raggedright
number\_project\strut
\end{minipage} & \begin{minipage}[t]{0.47\columnwidth}\raggedright
Number of projects employee contributes to\strut
\end{minipage}\tabularnewline
\begin{minipage}[t]{0.47\columnwidth}\raggedright
average\_monthly\_hours\strut
\end{minipage} & \begin{minipage}[t]{0.47\columnwidth}\raggedright
Average number of hours employee worked per month\strut
\end{minipage}\tabularnewline
\begin{minipage}[t]{0.47\columnwidth}\raggedright
time\_spend\_company\strut
\end{minipage} & \begin{minipage}[t]{0.47\columnwidth}\raggedright
How long the employee has been with the company (years)\strut
\end{minipage}\tabularnewline
\begin{minipage}[t]{0.47\columnwidth}\raggedright
Work\_accident\strut
\end{minipage} & \begin{minipage}[t]{0.47\columnwidth}\raggedright
Whether or not the employee experienced an accident while at work\strut
\end{minipage}\tabularnewline
\begin{minipage}[t]{0.47\columnwidth}\raggedright
left\strut
\end{minipage} & \begin{minipage}[t]{0.47\columnwidth}\raggedright
Whether or not the employee left the company\strut
\end{minipage}\tabularnewline
\begin{minipage}[t]{0.47\columnwidth}\raggedright
promotion\_last\_5years\strut
\end{minipage} & \begin{minipage}[t]{0.47\columnwidth}\raggedright
Whether or not the employee was promoted in the last 5 years\strut
\end{minipage}\tabularnewline
\begin{minipage}[t]{0.47\columnwidth}\raggedright
Department\strut
\end{minipage} & \begin{minipage}[t]{0.47\columnwidth}\raggedright
The employee's department\strut
\end{minipage}\tabularnewline
\begin{minipage}[t]{0.47\columnwidth}\raggedright
salary\strut
\end{minipage} & \begin{minipage}[t]{0.47\columnwidth}\raggedright
The employee's salary (U.S. dollars)\strut
\end{minipage}\tabularnewline
\bottomrule
\end{longtable}

    💭 \#\#\# Reflect on these questions as you complete the plan stage.

\begin{itemize}
\item
  Who are your stakeholders for this project? The stakeholders are the
  HR department at Salifort Motors, including HR managers and
  executives, who aim to improve employee retention based on data-driven
  insights.
\item
  What are you trying to solve or accomplish? The goal is to analyze
  employee data to identify factors contributing to turnover and build a
  predictive model to determine whether an employee is likely to leave.
  This will help HR implement targeted retention strategies.
\item
  What are your initial observations when you explore the data? The
  dataset contains 15,000 rows and 10 columns, including numerical
  (e.g., satisfaction\_level, average\_monthly\_hours) and categorical
  (e.g., Department, salary) variables. The target variable left is
  binary (0 = stayed, 1 = left). Initial exploration may reveal
  imbalances in the target variable or relationships between features
  like satisfaction and tenure.
\item
  What resources do you find yourself using as you complete this stage?
  (Make sure to include the links.) Kaggle dataset documentation: HR
  Analytics Dataset Pandas documentation: pandas.pydata.org Scikit-learn
  documentation: scikit-learn.org
\item
  Do you have any ethical considerations in this stage? Ethical
  considerations include ensuring employee data privacy, avoiding bias
  in model predictions (e.g., not discriminating based on department or
  salary), and transparently communicating model limitations to
  stakeholders.
\end{itemize}

    {[}Double-click to enter your responses here.{]}

    \hypertarget{step-1.-imports}{%
\subsection{Step 1. Imports}\label{step-1.-imports}}

\begin{itemize}
\tightlist
\item
  Import packages
\item
  Load dataset
\end{itemize}

    \hypertarget{import-packages}{%
\subsubsection{Import packages}\label{import-packages}}

    \begin{tcolorbox}[breakable, size=fbox, boxrule=1pt, pad at break*=1mm,colback=cellbackground, colframe=cellborder]
\prompt{In}{incolor}{2}{\boxspacing}
\begin{Verbatim}[commandchars=\\\{\}]
\PY{c+c1}{\PYZsh{} Import packages}
\PY{k+kn}{import} \PY{n+nn}{pandas} \PY{k}{as} \PY{n+nn}{pd}
\PY{k+kn}{import} \PY{n+nn}{numpy} \PY{k}{as} \PY{n+nn}{np}
\PY{k+kn}{import} \PY{n+nn}{matplotlib}\PY{n+nn}{.}\PY{n+nn}{pyplot} \PY{k}{as} \PY{n+nn}{plt}
\PY{k+kn}{import} \PY{n+nn}{seaborn} \PY{k}{as} \PY{n+nn}{sns}
\PY{k+kn}{from} \PY{n+nn}{sklearn}\PY{n+nn}{.}\PY{n+nn}{model\PYZus{}selection} \PY{k+kn}{import} \PY{n}{train\PYZus{}test\PYZus{}split}
\PY{k+kn}{from} \PY{n+nn}{sklearn}\PY{n+nn}{.}\PY{n+nn}{linear\PYZus{}model} \PY{k+kn}{import} \PY{n}{LogisticRegression}
\PY{k+kn}{from} \PY{n+nn}{sklearn}\PY{n+nn}{.}\PY{n+nn}{metrics} \PY{k+kn}{import} \PY{n}{accuracy\PYZus{}score}\PY{p}{,} \PY{n}{precision\PYZus{}score}\PY{p}{,} \PY{n}{recall\PYZus{}score}\PY{p}{,} \PY{n}{f1\PYZus{}score}\PY{p}{,} \PY{n}{roc\PYZus{}auc\PYZus{}score}\PY{p}{,} \PY{n}{roc\PYZus{}curve}
\PY{k+kn}{from} \PY{n+nn}{sklearn}\PY{n+nn}{.}\PY{n+nn}{preprocessing} \PY{k+kn}{import} \PY{n}{StandardScaler}
\end{Verbatim}
\end{tcolorbox}

    \hypertarget{load-dataset}{%
\subsubsection{Load dataset}\label{load-dataset}}

\texttt{Pandas} is used to read a dataset called
\textbf{\texttt{HR\_capstone\_dataset.csv}.} As shown in this cell, the
dataset has been automatically loaded in for you. You do not need to
download the .csv file, or provide more code, in order to access the
dataset and proceed with this lab. Please continue with this activity by
completing the following instructions.

    \begin{tcolorbox}[breakable, size=fbox, boxrule=1pt, pad at break*=1mm,colback=cellbackground, colframe=cellborder]
\prompt{In}{incolor}{3}{\boxspacing}
\begin{Verbatim}[commandchars=\\\{\}]
\PY{c+c1}{\PYZsh{} Load dataset}
\PY{n}{df0} \PY{o}{=} \PY{n}{pd}\PY{o}{.}\PY{n}{read\PYZus{}csv}\PY{p}{(}\PY{l+s+s2}{\PYZdq{}}\PY{l+s+s2}{HR\PYZus{}capstone\PYZus{}dataset.csv}\PY{l+s+s2}{\PYZdq{}}\PY{p}{)}

\PY{c+c1}{\PYZsh{} Display first few rows}
\PY{n}{df0}\PY{o}{.}\PY{n}{head}\PY{p}{(}\PY{p}{)}
\end{Verbatim}
\end{tcolorbox}

    \hypertarget{step-2.-data-exploration-initial-eda-and-data-cleaning}{%
\subsection{Step 2. Data Exploration (Initial EDA and data
cleaning)}\label{step-2.-data-exploration-initial-eda-and-data-cleaning}}

\begin{itemize}
\tightlist
\item
  Understand your variables
\item
  Clean your dataset (missing data, redundant data, outliers)
\end{itemize}

    \hypertarget{gather-basic-information-about-the-data}{%
\subsubsection{Gather basic information about the
data}\label{gather-basic-information-about-the-data}}

    \begin{tcolorbox}[breakable, size=fbox, boxrule=1pt, pad at break*=1mm,colback=cellbackground, colframe=cellborder]
\prompt{In}{incolor}{ }{\boxspacing}
\begin{Verbatim}[commandchars=\\\{\}]
\PY{c+c1}{\PYZsh{} Gather basic information about the data}
\PY{c+c1}{\PYZsh{} Basic information about the dataset}
\PY{n}{df0}\PY{o}{.}\PY{n}{info}\PY{p}{(}\PY{p}{)}
\end{Verbatim}
\end{tcolorbox}

    \hypertarget{gather-descriptive-statistics-about-the-data}{%
\subsubsection{Gather descriptive statistics about the
data}\label{gather-descriptive-statistics-about-the-data}}

    \begin{tcolorbox}[breakable, size=fbox, boxrule=1pt, pad at break*=1mm,colback=cellbackground, colframe=cellborder]
\prompt{In}{incolor}{ }{\boxspacing}
\begin{Verbatim}[commandchars=\\\{\}]
\PY{c+c1}{\PYZsh{} Gather descriptive statistics about the data}
\PY{c+c1}{\PYZsh{} Descriptive statistics}
\PY{n}{df0}\PY{o}{.}\PY{n}{describe}\PY{p}{(}\PY{p}{)}
\end{Verbatim}
\end{tcolorbox}

    \hypertarget{rename-columns}{%
\subsubsection{Rename columns}\label{rename-columns}}

    As a data cleaning step, rename the columns as needed. Standardize the
column names so that they are all in \texttt{snake\_case}, correct any
column names that are misspelled, and make column names more concise as
needed.

    \begin{tcolorbox}[breakable, size=fbox, boxrule=1pt, pad at break*=1mm,colback=cellbackground, colframe=cellborder]
\prompt{In}{incolor}{ }{\boxspacing}
\begin{Verbatim}[commandchars=\\\{\}]
\PY{c+c1}{\PYZsh{} Display all column names}
\PY{n+nb}{print}\PY{p}{(}\PY{n}{df0}\PY{o}{.}\PY{n}{columns}\PY{p}{)}
\end{Verbatim}
\end{tcolorbox}

    \begin{tcolorbox}[breakable, size=fbox, boxrule=1pt, pad at break*=1mm,colback=cellbackground, colframe=cellborder]
\prompt{In}{incolor}{ }{\boxspacing}
\begin{Verbatim}[commandchars=\\\{\}]
\PY{c+c1}{\PYZsh{} Rename columns as needed}
\PY{n}{df0} \PY{o}{=} \PY{n}{df0}\PY{o}{.}\PY{n}{rename}\PY{p}{(}\PY{n}{columns}\PY{o}{=}\PY{p}{\PYZob{}}
    \PY{l+s+s1}{\PYZsq{}}\PY{l+s+s1}{satisfaction\PYZus{}level}\PY{l+s+s1}{\PYZsq{}}\PY{p}{:} \PY{l+s+s1}{\PYZsq{}}\PY{l+s+s1}{satisfaction\PYZus{}level}\PY{l+s+s1}{\PYZsq{}}\PY{p}{,}
    \PY{l+s+s1}{\PYZsq{}}\PY{l+s+s1}{last\PYZus{}evaluation}\PY{l+s+s1}{\PYZsq{}}\PY{p}{:} \PY{l+s+s1}{\PYZsq{}}\PY{l+s+s1}{last\PYZus{}evaluation}\PY{l+s+s1}{\PYZsq{}}\PY{p}{,}
    \PY{l+s+s1}{\PYZsq{}}\PY{l+s+s1}{number\PYZus{}project}\PY{l+s+s1}{\PYZsq{}}\PY{p}{:} \PY{l+s+s1}{\PYZsq{}}\PY{l+s+s1}{number\PYZus{}of\PYZus{}projects}\PY{l+s+s1}{\PYZsq{}}\PY{p}{,}
    \PY{l+s+s1}{\PYZsq{}}\PY{l+s+s1}{average\PYZus{}montly\PYZus{}hours}\PY{l+s+s1}{\PYZsq{}}\PY{p}{:} \PY{l+s+s1}{\PYZsq{}}\PY{l+s+s1}{average\PYZus{}monthly\PYZus{}hours}\PY{l+s+s1}{\PYZsq{}}\PY{p}{,}  \PY{c+c1}{\PYZsh{} Correct typo if present}
    \PY{l+s+s1}{\PYZsq{}}\PY{l+s+s1}{time\PYZus{}spend\PYZus{}company}\PY{l+s+s1}{\PYZsq{}}\PY{p}{:} \PY{l+s+s1}{\PYZsq{}}\PY{l+s+s1}{tenure}\PY{l+s+s1}{\PYZsq{}}\PY{p}{,}
    \PY{l+s+s1}{\PYZsq{}}\PY{l+s+s1}{Work\PYZus{}accident}\PY{l+s+s1}{\PYZsq{}}\PY{p}{:} \PY{l+s+s1}{\PYZsq{}}\PY{l+s+s1}{work\PYZus{}accident}\PY{l+s+s1}{\PYZsq{}}\PY{p}{,}
    \PY{l+s+s1}{\PYZsq{}}\PY{l+s+s1}{left}\PY{l+s+s1}{\PYZsq{}}\PY{p}{:} \PY{l+s+s1}{\PYZsq{}}\PY{l+s+s1}{left}\PY{l+s+s1}{\PYZsq{}}\PY{p}{,}
    \PY{l+s+s1}{\PYZsq{}}\PY{l+s+s1}{promotion\PYZus{}last\PYZus{}5years}\PY{l+s+s1}{\PYZsq{}}\PY{p}{:} \PY{l+s+s1}{\PYZsq{}}\PY{l+s+s1}{promotion\PYZus{}last\PYZus{}5years}\PY{l+s+s1}{\PYZsq{}}\PY{p}{,}
    \PY{l+s+s1}{\PYZsq{}}\PY{l+s+s1}{Department}\PY{l+s+s1}{\PYZsq{}}\PY{p}{:} \PY{l+s+s1}{\PYZsq{}}\PY{l+s+s1}{department}\PY{l+s+s1}{\PYZsq{}}\PY{p}{,}
    \PY{l+s+s1}{\PYZsq{}}\PY{l+s+s1}{salary}\PY{l+s+s1}{\PYZsq{}}\PY{p}{:} \PY{l+s+s1}{\PYZsq{}}\PY{l+s+s1}{salary}\PY{l+s+s1}{\PYZsq{}}
\PY{p}{\PYZcb{}}\PY{p}{)}


\PY{c+c1}{\PYZsh{} Display all column names after the update}
\PY{n+nb}{print}\PY{p}{(}\PY{n}{df0}\PY{o}{.}\PY{n}{columns}\PY{p}{)}
\end{Verbatim}
\end{tcolorbox}

    \hypertarget{check-missing-values}{%
\subsubsection{Check missing values}\label{check-missing-values}}

    Check for any missing values in the data.

    \begin{tcolorbox}[breakable, size=fbox, boxrule=1pt, pad at break*=1mm,colback=cellbackground, colframe=cellborder]
\prompt{In}{incolor}{ }{\boxspacing}
\begin{Verbatim}[commandchars=\\\{\}]
\PY{c+c1}{\PYZsh{} Check for missing values}
\PY{n}{df0}\PY{o}{.}\PY{n}{isnull}\PY{p}{(}\PY{p}{)}\PY{o}{.}\PY{n}{sum}\PY{p}{(}\PY{p}{)}
\end{Verbatim}
\end{tcolorbox}

    \hypertarget{check-duplicates}{%
\subsubsection{Check duplicates}\label{check-duplicates}}

    Check for any duplicate entries in the data.

    \begin{tcolorbox}[breakable, size=fbox, boxrule=1pt, pad at break*=1mm,colback=cellbackground, colframe=cellborder]
\prompt{In}{incolor}{ }{\boxspacing}
\begin{Verbatim}[commandchars=\\\{\}]
\PY{c+c1}{\PYZsh{} Check for duplicates}
\PY{n+nb}{print}\PY{p}{(}\PY{n}{df0}\PY{o}{.}\PY{n}{duplicated}\PY{p}{(}\PY{p}{)}\PY{o}{.}\PY{n}{sum}\PY{p}{(}\PY{p}{)}\PY{p}{)}
\end{Verbatim}
\end{tcolorbox}

    \begin{tcolorbox}[breakable, size=fbox, boxrule=1pt, pad at break*=1mm,colback=cellbackground, colframe=cellborder]
\prompt{In}{incolor}{ }{\boxspacing}
\begin{Verbatim}[commandchars=\\\{\}]
\PY{c+c1}{\PYZsh{} Inspect some rows containing duplicates as needed}
\PY{n}{df0}\PY{p}{[}\PY{n}{df0}\PY{o}{.}\PY{n}{duplicated}\PY{p}{(}\PY{p}{)}\PY{p}{]}\PY{o}{.}\PY{n}{head}\PY{p}{(}\PY{p}{)}
\end{Verbatim}
\end{tcolorbox}

    \begin{tcolorbox}[breakable, size=fbox, boxrule=1pt, pad at break*=1mm,colback=cellbackground, colframe=cellborder]
\prompt{In}{incolor}{ }{\boxspacing}
\begin{Verbatim}[commandchars=\\\{\}]
\PY{c+c1}{\PYZsh{} Drop duplicates and save resulting dataframe in a new variable as needed}
\PY{n}{df1} \PY{o}{=} \PY{n}{df0}\PY{o}{.}\PY{n}{drop\PYZus{}duplicates}\PY{p}{(}\PY{p}{)}


\PY{c+c1}{\PYZsh{} Display first few rows of new dataframe as needed}
\PY{n+nb}{print}\PY{p}{(}\PY{l+s+sa}{f}\PY{l+s+s2}{\PYZdq{}}\PY{l+s+s2}{Original rows: }\PY{l+s+si}{\PYZob{}}\PY{n+nb}{len}\PY{p}{(}\PY{n}{df0}\PY{p}{)}\PY{l+s+si}{\PYZcb{}}\PY{l+s+s2}{, After dropping duplicates: }\PY{l+s+si}{\PYZob{}}\PY{n+nb}{len}\PY{p}{(}\PY{n}{df1}\PY{p}{)}\PY{l+s+si}{\PYZcb{}}\PY{l+s+s2}{\PYZdq{}}\PY{p}{)}
\PY{n}{df1}\PY{o}{.}\PY{n}{head}\PY{p}{(}\PY{p}{)}
\end{Verbatim}
\end{tcolorbox}

    \hypertarget{check-outliers}{%
\subsubsection{Check outliers}\label{check-outliers}}

    Check for outliers in the data.

    \begin{tcolorbox}[breakable, size=fbox, boxrule=1pt, pad at break*=1mm,colback=cellbackground, colframe=cellborder]
\prompt{In}{incolor}{ }{\boxspacing}
\begin{Verbatim}[commandchars=\\\{\}]
\PY{c+c1}{\PYZsh{} Create a boxplot to visualize distribution of `tenure` and detect any outliers}
\PY{n}{plt}\PY{o}{.}\PY{n}{figure}\PY{p}{(}\PY{n}{figsize}\PY{o}{=}\PY{p}{(}\PY{l+m+mi}{8}\PY{p}{,} \PY{l+m+mi}{4}\PY{p}{)}\PY{p}{)}
\PY{n}{sns}\PY{o}{.}\PY{n}{boxplot}\PY{p}{(}\PY{n}{x}\PY{o}{=}\PY{n}{df1}\PY{p}{[}\PY{l+s+s1}{\PYZsq{}}\PY{l+s+s1}{tenure}\PY{l+s+s1}{\PYZsq{}}\PY{p}{]}\PY{p}{)}
\PY{n}{plt}\PY{o}{.}\PY{n}{title}\PY{p}{(}\PY{l+s+s1}{\PYZsq{}}\PY{l+s+s1}{Boxplot of Tenure}\PY{l+s+s1}{\PYZsq{}}\PY{p}{)}
\PY{n}{plt}\PY{o}{.}\PY{n}{show}\PY{p}{(}\PY{p}{)}
\end{Verbatim}
\end{tcolorbox}

    \begin{tcolorbox}[breakable, size=fbox, boxrule=1pt, pad at break*=1mm,colback=cellbackground, colframe=cellborder]
\prompt{In}{incolor}{ }{\boxspacing}
\begin{Verbatim}[commandchars=\\\{\}]
\PY{c+c1}{\PYZsh{} Determine the number of rows containing outliers}
\PY{n}{Q1} \PY{o}{=} \PY{n}{df1}\PY{p}{[}\PY{l+s+s1}{\PYZsq{}}\PY{l+s+s1}{tenure}\PY{l+s+s1}{\PYZsq{}}\PY{p}{]}\PY{o}{.}\PY{n}{quantile}\PY{p}{(}\PY{l+m+mf}{0.25}\PY{p}{)}
\PY{n}{Q3} \PY{o}{=} \PY{n}{df1}\PY{p}{[}\PY{l+s+s1}{\PYZsq{}}\PY{l+s+s1}{tenure}\PY{l+s+s1}{\PYZsq{}}\PY{p}{]}\PY{o}{.}\PY{n}{quantile}\PY{p}{(}\PY{l+m+mf}{0.75}\PY{p}{)}
\PY{n}{IQR} \PY{o}{=} \PY{n}{Q3} \PY{o}{\PYZhy{}} \PY{n}{Q1}
\PY{n}{lower\PYZus{}bound} \PY{o}{=} \PY{n}{Q1} \PY{o}{\PYZhy{}} \PY{l+m+mf}{1.5} \PY{o}{*} \PY{n}{IQR}
\PY{n}{upper\PYZus{}bound} \PY{o}{=} \PY{n}{Q3} \PY{o}{+} \PY{l+m+mf}{1.5} \PY{o}{*} \PY{n}{IQR}
\PY{n}{outliers} \PY{o}{=} \PY{n}{df1}\PY{p}{[}\PY{p}{(}\PY{n}{df1}\PY{p}{[}\PY{l+s+s1}{\PYZsq{}}\PY{l+s+s1}{tenure}\PY{l+s+s1}{\PYZsq{}}\PY{p}{]} \PY{o}{\PYZlt{}} \PY{n}{lower\PYZus{}bound}\PY{p}{)} \PY{o}{|} \PY{p}{(}\PY{n}{df1}\PY{p}{[}\PY{l+s+s1}{\PYZsq{}}\PY{l+s+s1}{tenure}\PY{l+s+s1}{\PYZsq{}}\PY{p}{]} \PY{o}{\PYZgt{}} \PY{n}{upper\PYZus{}bound}\PY{p}{)}\PY{p}{]}
\PY{n+nb}{print}\PY{p}{(}\PY{l+s+sa}{f}\PY{l+s+s2}{\PYZdq{}}\PY{l+s+s2}{Number of outliers in tenure: }\PY{l+s+si}{\PYZob{}}\PY{n+nb}{len}\PY{p}{(}\PY{n}{outliers}\PY{p}{)}\PY{l+s+si}{\PYZcb{}}\PY{l+s+s2}{\PYZdq{}}\PY{p}{)}
\end{Verbatim}
\end{tcolorbox}

    Certain types of models are more sensitive to outliers than others. When
you get to the stage of building your model, consider whether to remove
outliers, based on the type of model you decide to use.

    \hypertarget{pace-analyze-stage}{%
\section{pAce: Analyze Stage}\label{pace-analyze-stage}}

\begin{itemize}
\tightlist
\item
  Perform EDA (analyze relationships between variables) \# Percentage of
  employees who left vs.~stayed print(df1{[}`left'{]}.value\_counts())
  print(df1{[}`left'{]}.value\_counts(normalize=True) * 100)
\end{itemize}

\hypertarget{visualizations}{%
\section{Visualizations}\label{visualizations}}

\hypertarget{satisfaction-level-vs.-left}{%
\section{1. Satisfaction level
vs.~left}\label{satisfaction-level-vs.-left}}

plt.figure(figsize=(8, 4)) sns.histplot(data=df1,
x=`satisfaction\_level', hue=`left', multiple=`stack')
plt.title(`Satisfaction Level Distribution by Turnover') plt.show()

\hypertarget{tenure-vs.-left}{%
\section{2. Tenure vs.~left}\label{tenure-vs.-left}}

plt.figure(figsize=(8, 4)) sns.boxplot(x=`left', y=`tenure', data=df1)
plt.title(`Tenure by Turnover') plt.show()

\hypertarget{correlation-heatmap-for-numerical-variables}{%
\section{3. Correlation heatmap for numerical
variables}\label{correlation-heatmap-for-numerical-variables}}

plt.figure(figsize=(10, 6))
sns.heatmap(df1.select\_dtypes(include={[}`float64', `int64'{]}).corr(),
annot=True, cmap=`coolwarm') plt.title(`Correlation Heatmap') plt.show()

    💭 \#\#\# Reflect on these questions as you complete the analyze stage.

\begin{itemize}
\item
  What did you observe about the relationships between variables? Lower
  satisfaction and moderate tenure (3--5 years) are associated with
  higher turnover. Salary and promotions may also play a role.
\item
  What do you observe about the distributions in the data? Satisfaction
  is left-skewed for employees who left. Tenure has outliers.
\item
  What transformations did you make with your data? Why did you chose to
  make those decisions? Dropped duplicates, renamed columns to
  snake\_case. May cap outliers or encode categorical variables later.
\item
  What are some purposes of EDA before constructing a predictive model?
  Understand variable relationships, identify cleaning needs, and inform
  feature selection for modeling.
\item
  What resources do you find yourself using as you complete this stage?
  (Make sure to include the links.) Same as Plan stage, plus Seaborn:
  seaborn.pydata.org.
\item
  Do you have any ethical considerations in this stage? Avoid
  overgeneralizing findings (e.g., assuming all low-satisfaction
  employees will leave) and ensure fair treatment across departments.
\end{itemize}

    

    \hypertarget{step-2.-data-exploration-continue-eda}{%
\subsection{Step 2. Data Exploration (Continue
EDA)}\label{step-2.-data-exploration-continue-eda}}

Begin by understanding how many employees left and what percentage of
all employees this figure represents.

    \begin{tcolorbox}[breakable, size=fbox, boxrule=1pt, pad at break*=1mm,colback=cellbackground, colframe=cellborder]
\prompt{In}{incolor}{ }{\boxspacing}
\begin{Verbatim}[commandchars=\\\{\}]
\PY{c+c1}{\PYZsh{} Gather basic information about the data}
\PY{n}{df0}\PY{o}{.}\PY{n}{info}\PY{p}{(}\PY{p}{)}
\end{Verbatim}
\end{tcolorbox}

    \hypertarget{data-visualizations}{%
\subsubsection{Data visualizations}\label{data-visualizations}}

    Now, examine variables that you're interested in, and create plots to
visualize relationships between variables in the data.

    \begin{tcolorbox}[breakable, size=fbox, boxrule=1pt, pad at break*=1mm,colback=cellbackground, colframe=cellborder]
\prompt{In}{incolor}{ }{\boxspacing}
\begin{Verbatim}[commandchars=\\\{\}]
\PY{c+c1}{\PYZsh{} Visualization 1: Satisfaction Level Distribution by Turnover}
\PY{n}{plt}\PY{o}{.}\PY{n}{figure}\PY{p}{(}\PY{n}{figsize}\PY{o}{=}\PY{p}{(}\PY{l+m+mi}{8}\PY{p}{,} \PY{l+m+mi}{5}\PY{p}{)}\PY{p}{)}
\PY{n}{sns}\PY{o}{.}\PY{n}{histplot}\PY{p}{(}\PY{n}{data}\PY{o}{=}\PY{n}{df1}\PY{p}{,} \PY{n}{x}\PY{o}{=}\PY{l+s+s1}{\PYZsq{}}\PY{l+s+s1}{satisfaction\PYZus{}level}\PY{l+s+s1}{\PYZsq{}}\PY{p}{,} \PY{n}{hue}\PY{o}{=}\PY{l+s+s1}{\PYZsq{}}\PY{l+s+s1}{left}\PY{l+s+s1}{\PYZsq{}}\PY{p}{,} \PY{n}{multiple}\PY{o}{=}\PY{l+s+s1}{\PYZsq{}}\PY{l+s+s1}{stack}\PY{l+s+s1}{\PYZsq{}}\PY{p}{,} \PY{n}{bins}\PY{o}{=}\PY{l+m+mi}{30}\PY{p}{)}
\PY{n}{plt}\PY{o}{.}\PY{n}{title}\PY{p}{(}\PY{l+s+s1}{\PYZsq{}}\PY{l+s+s1}{Satisfaction Level Distribution by Turnover}\PY{l+s+s1}{\PYZsq{}}\PY{p}{)}
\PY{n}{plt}\PY{o}{.}\PY{n}{xlabel}\PY{p}{(}\PY{l+s+s1}{\PYZsq{}}\PY{l+s+s1}{Satisfaction Level}\PY{l+s+s1}{\PYZsq{}}\PY{p}{)}
\PY{n}{plt}\PY{o}{.}\PY{n}{ylabel}\PY{p}{(}\PY{l+s+s1}{\PYZsq{}}\PY{l+s+s1}{Count}\PY{l+s+s1}{\PYZsq{}}\PY{p}{)}
\PY{n}{plt}\PY{o}{.}\PY{n}{legend}\PY{p}{(}\PY{n}{title}\PY{o}{=}\PY{l+s+s1}{\PYZsq{}}\PY{l+s+s1}{Left}\PY{l+s+s1}{\PYZsq{}}\PY{p}{,} \PY{n}{labels}\PY{o}{=}\PY{p}{[}\PY{l+s+s1}{\PYZsq{}}\PY{l+s+s1}{Stayed}\PY{l+s+s1}{\PYZsq{}}\PY{p}{,} \PY{l+s+s1}{\PYZsq{}}\PY{l+s+s1}{Left}\PY{l+s+s1}{\PYZsq{}}\PY{p}{]}\PY{p}{)}
\PY{n}{plt}\PY{o}{.}\PY{n}{show}\PY{p}{(}\PY{p}{)}
\end{Verbatim}
\end{tcolorbox}

    \begin{tcolorbox}[breakable, size=fbox, boxrule=1pt, pad at break*=1mm,colback=cellbackground, colframe=cellborder]
\prompt{In}{incolor}{ }{\boxspacing}
\begin{Verbatim}[commandchars=\\\{\}]
\PY{c+c1}{\PYZsh{} Visualization 2: Tenure vs. Turnover (Boxplot)}
\PY{n}{plt}\PY{o}{.}\PY{n}{figure}\PY{p}{(}\PY{n}{figsize}\PY{o}{=}\PY{p}{(}\PY{l+m+mi}{8}\PY{p}{,} \PY{l+m+mi}{5}\PY{p}{)}\PY{p}{)}
\PY{n}{sns}\PY{o}{.}\PY{n}{boxplot}\PY{p}{(}\PY{n}{x}\PY{o}{=}\PY{l+s+s1}{\PYZsq{}}\PY{l+s+s1}{left}\PY{l+s+s1}{\PYZsq{}}\PY{p}{,} \PY{n}{y}\PY{o}{=}\PY{l+s+s1}{\PYZsq{}}\PY{l+s+s1}{tenure}\PY{l+s+s1}{\PYZsq{}}\PY{p}{,} \PY{n}{data}\PY{o}{=}\PY{n}{df1}\PY{p}{)}
\PY{n}{plt}\PY{o}{.}\PY{n}{title}\PY{p}{(}\PY{l+s+s1}{\PYZsq{}}\PY{l+s+s1}{Tenure by Turnover}\PY{l+s+s1}{\PYZsq{}}\PY{p}{)}
\PY{n}{plt}\PY{o}{.}\PY{n}{xlabel}\PY{p}{(}\PY{l+s+s1}{\PYZsq{}}\PY{l+s+s1}{Left (0 = Stayed, 1 = Left)}\PY{l+s+s1}{\PYZsq{}}\PY{p}{)}
\PY{n}{plt}\PY{o}{.}\PY{n}{ylabel}\PY{p}{(}\PY{l+s+s1}{\PYZsq{}}\PY{l+s+s1}{Tenure (Years)}\PY{l+s+s1}{\PYZsq{}}\PY{p}{)}
\PY{n}{plt}\PY{o}{.}\PY{n}{xticks}\PY{p}{(}\PY{p}{[}\PY{l+m+mi}{0}\PY{p}{,} \PY{l+m+mi}{1}\PY{p}{]}\PY{p}{,} \PY{p}{[}\PY{l+s+s1}{\PYZsq{}}\PY{l+s+s1}{Stayed}\PY{l+s+s1}{\PYZsq{}}\PY{p}{,} \PY{l+s+s1}{\PYZsq{}}\PY{l+s+s1}{Left}\PY{l+s+s1}{\PYZsq{}}\PY{p}{]}\PY{p}{)}
\PY{n}{plt}\PY{o}{.}\PY{n}{show}\PY{p}{(}\PY{p}{)}
\end{Verbatim}
\end{tcolorbox}

    \begin{tcolorbox}[breakable, size=fbox, boxrule=1pt, pad at break*=1mm,colback=cellbackground, colframe=cellborder]
\prompt{In}{incolor}{ }{\boxspacing}
\begin{Verbatim}[commandchars=\\\{\}]
\PY{c+c1}{\PYZsh{} Visualization 3: Average Monthly Hours vs. Turnover (Boxplot)}
\PY{n}{plt}\PY{o}{.}\PY{n}{figure}\PY{p}{(}\PY{n}{figsize}\PY{o}{=}\PY{p}{(}\PY{l+m+mi}{8}\PY{p}{,} \PY{l+m+mi}{5}\PY{p}{)}\PY{p}{)}
\PY{n}{sns}\PY{o}{.}\PY{n}{boxplot}\PY{p}{(}\PY{n}{x}\PY{o}{=}\PY{l+s+s1}{\PYZsq{}}\PY{l+s+s1}{left}\PY{l+s+s1}{\PYZsq{}}\PY{p}{,} \PY{n}{y}\PY{o}{=}\PY{l+s+s1}{\PYZsq{}}\PY{l+s+s1}{average\PYZus{}monthly\PYZus{}hours}\PY{l+s+s1}{\PYZsq{}}\PY{p}{,} \PY{n}{data}\PY{o}{=}\PY{n}{df1}\PY{p}{)}
\PY{n}{plt}\PY{o}{.}\PY{n}{title}\PY{p}{(}\PY{l+s+s1}{\PYZsq{}}\PY{l+s+s1}{Average Monthly Hours by Turnover}\PY{l+s+s1}{\PYZsq{}}\PY{p}{)}
\PY{n}{plt}\PY{o}{.}\PY{n}{xlabel}\PY{p}{(}\PY{l+s+s1}{\PYZsq{}}\PY{l+s+s1}{Left (0 = Stayed, 1 = Left)}\PY{l+s+s1}{\PYZsq{}}\PY{p}{)}
\PY{n}{plt}\PY{o}{.}\PY{n}{ylabel}\PY{p}{(}\PY{l+s+s1}{\PYZsq{}}\PY{l+s+s1}{Average Monthly Hours}\PY{l+s+s1}{\PYZsq{}}\PY{p}{)}
\PY{n}{plt}\PY{o}{.}\PY{n}{xticks}\PY{p}{(}\PY{p}{[}\PY{l+m+mi}{0}\PY{p}{,} \PY{l+m+mi}{1}\PY{p}{]}\PY{p}{,} \PY{p}{[}\PY{l+s+s1}{\PYZsq{}}\PY{l+s+s1}{Stayed}\PY{l+s+s1}{\PYZsq{}}\PY{p}{,} \PY{l+s+s1}{\PYZsq{}}\PY{l+s+s1}{Left}\PY{l+s+s1}{\PYZsq{}}\PY{p}{]}\PY{p}{)}
\PY{n}{plt}\PY{o}{.}\PY{n}{show}\PY{p}{(}\PY{p}{)}
\end{Verbatim}
\end{tcolorbox}

    \begin{tcolorbox}[breakable, size=fbox, boxrule=1pt, pad at break*=1mm,colback=cellbackground, colframe=cellborder]
\prompt{In}{incolor}{ }{\boxspacing}
\begin{Verbatim}[commandchars=\\\{\}]
\PY{c+c1}{\PYZsh{} Visualization 4: Turnover by Salary Level (Countplot)}
\PY{n}{plt}\PY{o}{.}\PY{n}{figure}\PY{p}{(}\PY{n}{figsize}\PY{o}{=}\PY{p}{(}\PY{l+m+mi}{8}\PY{p}{,} \PY{l+m+mi}{5}\PY{p}{)}\PY{p}{)}
\PY{n}{sns}\PY{o}{.}\PY{n}{countplot}\PY{p}{(}\PY{n}{x}\PY{o}{=}\PY{l+s+s1}{\PYZsq{}}\PY{l+s+s1}{salary}\PY{l+s+s1}{\PYZsq{}}\PY{p}{,} \PY{n}{hue}\PY{o}{=}\PY{l+s+s1}{\PYZsq{}}\PY{l+s+s1}{left}\PY{l+s+s1}{\PYZsq{}}\PY{p}{,} \PY{n}{data}\PY{o}{=}\PY{n}{df1}\PY{p}{)}
\PY{n}{plt}\PY{o}{.}\PY{n}{title}\PY{p}{(}\PY{l+s+s1}{\PYZsq{}}\PY{l+s+s1}{Turnover by Salary Level}\PY{l+s+s1}{\PYZsq{}}\PY{p}{)}
\PY{n}{plt}\PY{o}{.}\PY{n}{xlabel}\PY{p}{(}\PY{l+s+s1}{\PYZsq{}}\PY{l+s+s1}{Salary Level}\PY{l+s+s1}{\PYZsq{}}\PY{p}{)}
\PY{n}{plt}\PY{o}{.}\PY{n}{ylabel}\PY{p}{(}\PY{l+s+s1}{\PYZsq{}}\PY{l+s+s1}{Count}\PY{l+s+s1}{\PYZsq{}}\PY{p}{)}
\PY{n}{plt}\PY{o}{.}\PY{n}{legend}\PY{p}{(}\PY{n}{title}\PY{o}{=}\PY{l+s+s1}{\PYZsq{}}\PY{l+s+s1}{Left}\PY{l+s+s1}{\PYZsq{}}\PY{p}{,} \PY{n}{labels}\PY{o}{=}\PY{p}{[}\PY{l+s+s1}{\PYZsq{}}\PY{l+s+s1}{Stayed}\PY{l+s+s1}{\PYZsq{}}\PY{p}{,} \PY{l+s+s1}{\PYZsq{}}\PY{l+s+s1}{Left}\PY{l+s+s1}{\PYZsq{}}\PY{p}{]}\PY{p}{)}
\PY{n}{plt}\PY{o}{.}\PY{n}{show}\PY{p}{(}\PY{p}{)}
\end{Verbatim}
\end{tcolorbox}

    \begin{tcolorbox}[breakable, size=fbox, boxrule=1pt, pad at break*=1mm,colback=cellbackground, colframe=cellborder]
\prompt{In}{incolor}{ }{\boxspacing}
\begin{Verbatim}[commandchars=\\\{\}]
\PY{c+c1}{\PYZsh{} Visualization 5: Turnover by Promotion in Last 5 Years (Countplot)}
\PY{n}{plt}\PY{o}{.}\PY{n}{figure}\PY{p}{(}\PY{n}{figsize}\PY{o}{=}\PY{p}{(}\PY{l+m+mi}{8}\PY{p}{,} \PY{l+m+mi}{5}\PY{p}{)}\PY{p}{)}
\PY{n}{sns}\PY{o}{.}\PY{n}{countplot}\PY{p}{(}\PY{n}{x}\PY{o}{=}\PY{l+s+s1}{\PYZsq{}}\PY{l+s+s1}{promotion\PYZus{}last\PYZus{}5years}\PY{l+s+s1}{\PYZsq{}}\PY{p}{,} \PY{n}{hue}\PY{o}{=}\PY{l+s+s1}{\PYZsq{}}\PY{l+s+s1}{left}\PY{l+s+s1}{\PYZsq{}}\PY{p}{,} \PY{n}{data}\PY{o}{=}\PY{n}{df1}\PY{p}{)}
\PY{n}{plt}\PY{o}{.}\PY{n}{title}\PY{p}{(}\PY{l+s+s1}{\PYZsq{}}\PY{l+s+s1}{Turnover by Promotion in Last 5 Years}\PY{l+s+s1}{\PYZsq{}}\PY{p}{)}
\PY{n}{plt}\PY{o}{.}\PY{n}{xlabel}\PY{p}{(}\PY{l+s+s1}{\PYZsq{}}\PY{l+s+s1}{Promotion in Last 5 Years (0 = No, 1 = Yes)}\PY{l+s+s1}{\PYZsq{}}\PY{p}{)}
\PY{n}{plt}\PY{o}{.}\PY{n}{ylabel}\PY{p}{(}\PY{l+s+s1}{\PYZsq{}}\PY{l+s+s1}{Count}\PY{l+s+s1}{\PYZsq{}}\PY{p}{)}
\PY{n}{plt}\PY{o}{.}\PY{n}{legend}\PY{p}{(}\PY{n}{title}\PY{o}{=}\PY{l+s+s1}{\PYZsq{}}\PY{l+s+s1}{Left}\PY{l+s+s1}{\PYZsq{}}\PY{p}{,} \PY{n}{labels}\PY{o}{=}\PY{p}{[}\PY{l+s+s1}{\PYZsq{}}\PY{l+s+s1}{Stayed}\PY{l+s+s1}{\PYZsq{}}\PY{p}{,} \PY{l+s+s1}{\PYZsq{}}\PY{l+s+s1}{Left}\PY{l+s+s1}{\PYZsq{}}\PY{p}{]}\PY{p}{)}
\PY{n}{plt}\PY{o}{.}\PY{n}{xticks}\PY{p}{(}\PY{p}{[}\PY{l+m+mi}{0}\PY{p}{,} \PY{l+m+mi}{1}\PY{p}{]}\PY{p}{,} \PY{p}{[}\PY{l+s+s1}{\PYZsq{}}\PY{l+s+s1}{No}\PY{l+s+s1}{\PYZsq{}}\PY{p}{,} \PY{l+s+s1}{\PYZsq{}}\PY{l+s+s1}{Yes}\PY{l+s+s1}{\PYZsq{}}\PY{p}{]}\PY{p}{)}
\PY{n}{plt}\PY{o}{.}\PY{n}{show}\PY{p}{(}\PY{p}{)}
\end{Verbatim}
\end{tcolorbox}

    \begin{tcolorbox}[breakable, size=fbox, boxrule=1pt, pad at break*=1mm,colback=cellbackground, colframe=cellborder]
\prompt{In}{incolor}{ }{\boxspacing}
\begin{Verbatim}[commandchars=\\\{\}]
\PY{c+c1}{\PYZsh{} Visualization 6: Correlation Heatmap for Numerical Variables}
\PY{n}{plt}\PY{o}{.}\PY{n}{figure}\PY{p}{(}\PY{n}{figsize}\PY{o}{=}\PY{p}{(}\PY{l+m+mi}{10}\PY{p}{,} \PY{l+m+mi}{6}\PY{p}{)}\PY{p}{)}
\PY{n}{numerical\PYZus{}cols} \PY{o}{=} \PY{n}{df1}\PY{o}{.}\PY{n}{select\PYZus{}dtypes}\PY{p}{(}\PY{n}{include}\PY{o}{=}\PY{p}{[}\PY{l+s+s1}{\PYZsq{}}\PY{l+s+s1}{float64}\PY{l+s+s1}{\PYZsq{}}\PY{p}{,} \PY{l+s+s1}{\PYZsq{}}\PY{l+s+s1}{int64}\PY{l+s+s1}{\PYZsq{}}\PY{p}{]}\PY{p}{)}\PY{o}{.}\PY{n}{columns}
\PY{n}{sns}\PY{o}{.}\PY{n}{heatmap}\PY{p}{(}\PY{n}{df1}\PY{p}{[}\PY{n}{numerical\PYZus{}cols}\PY{p}{]}\PY{o}{.}\PY{n}{corr}\PY{p}{(}\PY{p}{)}\PY{p}{,} \PY{n}{annot}\PY{o}{=}\PY{k+kc}{True}\PY{p}{,} \PY{n}{cmap}\PY{o}{=}\PY{l+s+s1}{\PYZsq{}}\PY{l+s+s1}{coolwarm}\PY{l+s+s1}{\PYZsq{}}\PY{p}{,} \PY{n}{vmin}\PY{o}{=}\PY{o}{\PYZhy{}}\PY{l+m+mi}{1}\PY{p}{,} \PY{n}{vmax}\PY{o}{=}\PY{l+m+mi}{1}\PY{p}{)}
\PY{n}{plt}\PY{o}{.}\PY{n}{title}\PY{p}{(}\PY{l+s+s1}{\PYZsq{}}\PY{l+s+s1}{Correlation Heatmap of Numerical Variables}\PY{l+s+s1}{\PYZsq{}}\PY{p}{)}
\PY{n}{plt}\PY{o}{.}\PY{n}{show}\PY{p}{(}\PY{p}{)}
\end{Verbatim}
\end{tcolorbox}

    \begin{tcolorbox}[breakable, size=fbox, boxrule=1pt, pad at break*=1mm,colback=cellbackground, colframe=cellborder]
\prompt{In}{incolor}{ }{\boxspacing}
\begin{Verbatim}[commandchars=\\\{\}]
\PY{c+c1}{\PYZsh{} Create a plot as needed}
\PY{c+c1}{\PYZsh{}\PYZsh{}\PYZsh{} YOUR CODE HERE \PYZsh{}\PYZsh{}\PYZsh{}}
\end{Verbatim}
\end{tcolorbox}

    \begin{tcolorbox}[breakable, size=fbox, boxrule=1pt, pad at break*=1mm,colback=cellbackground, colframe=cellborder]
\prompt{In}{incolor}{ }{\boxspacing}
\begin{Verbatim}[commandchars=\\\{\}]
\PY{c+c1}{\PYZsh{} Create a plot as needed}
\PY{c+c1}{\PYZsh{}\PYZsh{}\PYZsh{} YOUR CODE HERE \PYZsh{}\PYZsh{}\PYZsh{}}
\end{Verbatim}
\end{tcolorbox}

    \hypertarget{insights}{%
\subsubsection{Insights}\label{insights}}

    Satisfaction Level by Turnover (Histogram) Observation: Employees who
left (left = 1) have a peak in satisfaction\_level around 0.4, while
those who stayed (left = 0) show higher satisfaction levels (mostly
above 0.6). Insight: Low satisfaction is a strong indicator of turnover.
HR should focus on improving employee satisfaction through engagement
initiatives or addressing workplace concerns. Tenure by Turnover
(Boxplot) Observation: Employees who left have a median tenure of
\textasciitilde4 years (range 3--5 years), while those who stayed have a
broader range (2--6 years). Insight: The 3--5-year tenure period is a
critical risk window for turnover, possibly due to stagnation or lack of
advancement. Targeted retention strategies for mid-tenure employees are
needed. Average Monthly Hours by Turnover (Boxplot) Observation:
Employees who left have extreme workloads---either very low
(\textasciitilde150 hours) or very high (\textasciitilde250+
hours)---compared to a median of \textasciitilde200 hours for those who
stayed. Insight: Both overworked and underutilized employees are more
likely to leave. HR should balance workloads to prevent burnout or
disengagement. Turnover by Salary Level (Countplot) Observation:
Turnover is highest among low-salary employees, followed by
medium-salary, with high-salary employees showing the lowest turnover.
Insight: Low compensation is a significant driver of turnover. Reviewing
salary structures for low- and medium-salary employees could improve
retention. Turnover by Number of Projects (Countplot) Observation:
Employees with 2 projects or 6--7 projects have higher turnover rates
compared to those with 3--5 projects. Insight: Both under-assignment (2
projects) and over-assignment (6--7 projects) are associated with
turnover, suggesting that optimal project loads (3--5) may enhance
retention. HR should monitor project assignments to avoid extremes.

    \hypertarget{pace-construct-stage}{%
\section{paCe: Construct Stage}\label{pace-construct-stage}}

\begin{itemize}
\tightlist
\item
  Determine which models are most appropriate
\item
  Construct the model
\item
  Confirm model assumptions
\item
  Evaluate model results to determine how well your model fits the data
\end{itemize}

    🔎 \#\# Recall model assumptions

\textbf{Logistic Regression model assumptions} - Outcome variable is
categorical - Observations are independent of each other - No severe
multicollinearity among X variables - No extreme outliers - Linear
relationship between each X variable and the logit of the outcome
variable - Sufficiently large sample size

    💭

    {[}Double-click to enter your responses here.{]}

    \hypertarget{step-3.-model-building-step-4.-results-and-evaluation}{%
\subsection{Step 3. Model Building, Step 4. Results and
Evaluation}\label{step-3.-model-building-step-4.-results-and-evaluation}}

\begin{itemize}
\tightlist
\item
  Fit a model that predicts the outcome variable using two or more
  independent variables
\item
  Check model assumptions
\item
  Evaluate the model
\end{itemize}

    \hypertarget{identify-the-type-of-prediction-task.}{%
\subsubsection{Identify the type of prediction
task.}\label{identify-the-type-of-prediction-task.}}

    The prediction task is binary classification, as the outcome variable
left is categorical with two classes: 0 (employee stayed) or 1 (employee
left). The goal is to predict whether an employee will leave based on
features like satisfaction\_level, tenure, average\_monthly\_hours, etc.

    \hypertarget{identify-the-types-of-models-most-appropriate-for-this-task.}{%
\subsubsection{Identify the types of models most appropriate for this
task.}\label{identify-the-types-of-models-most-appropriate-for-this-task.}}

    The most appropriate models for binary classification include:

Logistic Regression: Suitable due to the binary outcome,
interpretability of coefficients (important for HR stakeholders), and
the dataset's size (\textasciitilde11,991 rows post-duplicates). It
assumes linear relationships between features and the logit of the
outcome. Random Forest: A robust machine learning model that handles
non-linear relationships, interactions, and imbalanced data (turnover
rate \textasciitilde16.6\%). It's less interpretable but potentially
more accurate. Support Vector Machine (SVM): Effective for binary
classification, especially with scaled features, but computationally
intensive and less interpretable. Gradient Boosting (e.g., XGBoost):
Handles imbalanced data well and captures complex patterns, but requires
tuning and is less interpretable. Chosen Model: Logistic regression is
selected for this project because:

It provides interpretable coefficients to identify turnover drivers
(e.g., satisfaction, salary). It aligns with the project's focus on
regression models and the dataset's characteristics (large sample, no
severe multicollinearity from Step 2's heatmap). It's computationally
efficient for initial modeling.

    \hypertarget{modeling}{%
\subsubsection{Modeling}\label{modeling}}

Add as many cells as you need to conduct the modeling process.

    \begin{tcolorbox}[breakable, size=fbox, boxrule=1pt, pad at break*=1mm,colback=cellbackground, colframe=cellborder]
\prompt{In}{incolor}{ }{\boxspacing}
\begin{Verbatim}[commandchars=\\\{\}]
\PY{c+c1}{\PYZsh{} Import required libraries (assumed imported in Step 1, included for clarity)}
\PY{k+kn}{import} \PY{n+nn}{pandas} \PY{k}{as} \PY{n+nn}{pd}
\PY{k+kn}{import} \PY{n+nn}{numpy} \PY{k}{as} \PY{n+nn}{np}
\PY{k+kn}{import} \PY{n+nn}{matplotlib}\PY{n+nn}{.}\PY{n+nn}{pyplot} \PY{k}{as} \PY{n+nn}{plt}
\PY{k+kn}{import} \PY{n+nn}{seaborn} \PY{k}{as} \PY{n+nn}{sns}
\PY{k+kn}{from} \PY{n+nn}{sklearn}\PY{n+nn}{.}\PY{n+nn}{model\PYZus{}selection} \PY{k+kn}{import} \PY{n}{train\PYZus{}test\PYZus{}split}
\PY{k+kn}{from} \PY{n+nn}{sklearn}\PY{n+nn}{.}\PY{n+nn}{linear\PYZus{}model} \PY{k+kn}{import} \PY{n}{LogisticRegression}
\PY{k+kn}{from} \PY{n+nn}{sklearn}\PY{n+nn}{.}\PY{n+nn}{metrics} \PY{k+kn}{import} \PY{n}{accuracy\PYZus{}score}\PY{p}{,} \PY{n}{precision\PYZus{}score}\PY{p}{,} \PY{n}{recall\PYZus{}score}\PY{p}{,} \PY{n}{f1\PYZus{}score}\PY{p}{,} \PY{n}{roc\PYZus{}auc\PYZus{}score}\PY{p}{,} \PY{n}{roc\PYZus{}curve}
\PY{k+kn}{from} \PY{n+nn}{sklearn}\PY{n+nn}{.}\PY{n+nn}{preprocessing} \PY{k+kn}{import} \PY{n}{StandardScaler}

\PY{c+c1}{\PYZsh{} Assuming df1 is the cleaned DataFrame from Step 2 (duplicates dropped, columns renamed)}

\PY{c+c1}{\PYZsh{} Encode categorical variables}
\PY{n}{df\PYZus{}model} \PY{o}{=} \PY{n}{pd}\PY{o}{.}\PY{n}{get\PYZus{}dummies}\PY{p}{(}\PY{n}{df1}\PY{p}{,} \PY{n}{columns}\PY{o}{=}\PY{p}{[}\PY{l+s+s1}{\PYZsq{}}\PY{l+s+s1}{department}\PY{l+s+s1}{\PYZsq{}}\PY{p}{,} \PY{l+s+s1}{\PYZsq{}}\PY{l+s+s1}{salary}\PY{l+s+s1}{\PYZsq{}}\PY{p}{]}\PY{p}{,} \PY{n}{drop\PYZus{}first}\PY{o}{=}\PY{k+kc}{True}\PY{p}{)}

\PY{c+c1}{\PYZsh{} Define features (X) and target (y)}
\PY{n}{X} \PY{o}{=} \PY{n}{df\PYZus{}model}\PY{o}{.}\PY{n}{drop}\PY{p}{(}\PY{l+s+s1}{\PYZsq{}}\PY{l+s+s1}{left}\PY{l+s+s1}{\PYZsq{}}\PY{p}{,} \PY{n}{axis}\PY{o}{=}\PY{l+m+mi}{1}\PY{p}{)}
\PY{n}{y} \PY{o}{=} \PY{n}{df\PYZus{}model}\PY{p}{[}\PY{l+s+s1}{\PYZsq{}}\PY{l+s+s1}{left}\PY{l+s+s1}{\PYZsq{}}\PY{p}{]}

\PY{c+c1}{\PYZsh{} Split data into training (80\PYZpc{}) and testing (20\PYZpc{}) sets}
\PY{n}{X\PYZus{}train}\PY{p}{,} \PY{n}{X\PYZus{}test}\PY{p}{,} \PY{n}{y\PYZus{}train}\PY{p}{,} \PY{n}{y\PYZus{}test} \PY{o}{=} \PY{n}{train\PYZus{}test\PYZus{}split}\PY{p}{(}\PY{n}{X}\PY{p}{,} \PY{n}{y}\PY{p}{,} \PY{n}{test\PYZus{}size}\PY{o}{=}\PY{l+m+mf}{0.2}\PY{p}{,} \PY{n}{random\PYZus{}state}\PY{o}{=}\PY{l+m+mi}{42}\PY{p}{)}

\PY{c+c1}{\PYZsh{} Scale numerical features}
\PY{n}{scaler} \PY{o}{=} \PY{n}{StandardScaler}\PY{p}{(}\PY{p}{)}
\PY{n}{X\PYZus{}train\PYZus{}scaled} \PY{o}{=} \PY{n}{scaler}\PY{o}{.}\PY{n}{fit\PYZus{}transform}\PY{p}{(}\PY{n}{X\PYZus{}train}\PY{p}{)}
\PY{n}{X\PYZus{}test\PYZus{}scaled} \PY{o}{=} \PY{n}{scaler}\PY{o}{.}\PY{n}{transform}\PY{p}{(}\PY{n}{X\PYZus{}test}\PY{p}{)}

\PY{c+c1}{\PYZsh{} Fit logistic regression model}
\PY{n}{model} \PY{o}{=} \PY{n}{LogisticRegression}\PY{p}{(}\PY{n}{max\PYZus{}iter}\PY{o}{=}\PY{l+m+mi}{1000}\PY{p}{,} \PY{n}{random\PYZus{}state}\PY{o}{=}\PY{l+m+mi}{42}\PY{p}{)}
\PY{n}{model}\PY{o}{.}\PY{n}{fit}\PY{p}{(}\PY{n}{X\PYZus{}train\PYZus{}scaled}\PY{p}{,} \PY{n}{y\PYZus{}train}\PY{p}{)}

\PY{c+c1}{\PYZsh{} Make predictions}
\PY{n}{y\PYZus{}pred} \PY{o}{=} \PY{n}{model}\PY{o}{.}\PY{n}{predict}\PY{p}{(}\PY{n}{X\PYZus{}test\PYZus{}scaled}\PY{p}{)}
\end{Verbatim}
\end{tcolorbox}

    \hypertarget{pace-execute-stage}{%
\section{pacE: Execute Stage}\label{pace-execute-stage}}

\begin{itemize}
\tightlist
\item
  Interpret model performance and results
\item
  Share actionable steps with stakeholders
\end{itemize}

    ✏ \#\# Recall evaluation metrics

\begin{itemize}
\tightlist
\item
  \textbf{AUC} is the area under the ROC curve; it's also considered the
  probability that the model ranks a random positive example more highly
  than a random negative example.
\item
  \textbf{Precision} measures the proportion of data points predicted as
  True that are actually True, in other words, the proportion of
  positive predictions that are true positives.
\item
  \textbf{Recall} measures the proportion of data points that are
  predicted as True, out of all the data points that are actually True.
  In other words, it measures the proportion of positives that are
  correctly classified.
\item
  \textbf{Accuracy} measures the proportion of data points that are
  correctly classified.
\item
  \textbf{F1-score} is an aggregation of precision and recall.
\end{itemize}

    💭 \#\#\# Reflect on these questions as you complete the executing stage.

\begin{itemize}
\tightlist
\item
  What key insights emerged from your model(s)?
\item
  What business recommendations do you propose based on the models
  built?
\item
  What potential recommendations would you make to your manager/company?
\item
  Do you think your model could be improved? Why or why not? How?
\item
  Given what you know about the data and the models you were using, what
  other questions could you address for the team?
\item
  What resources do you find yourself using as you complete this stage?
  (Make sure to include the links.)
\item
  Do you have any ethical considerations in this stage?
\end{itemize}

    Double-click to enter your responses here.

    \hypertarget{step-4.-results-and-evaluation}{%
\subsection{Step 4. Results and
Evaluation}\label{step-4.-results-and-evaluation}}

\begin{itemize}
\tightlist
\item
  Interpret model
\item
  Evaluate model performance using metrics
\item
  Prepare results, visualizations, and actionable steps to share with
  stakeholders
\end{itemize}

Interpret Model The logistic regression model predicts whether an
employee will leave Salifort Motors (left: 0 = stayed, 1 = left) based
on features like satisfaction\_level, tenure, average\_monthly\_hours,
number\_of\_projects, work\_accident, promotion\_last\_5years,
department, and salary. The model's coefficients indicate the direction
and strength of each feature's impact on turnover likelihood:

Negative coefficients (e.g., satisfaction\_level,
promotion\_last\_5years, salary\_high): Higher values decrease the
probability of leaving. Positive coefficients (e.g., tenure,
average\_monthly\_hours): Higher values increase the probability of
leaving. Evaluate Model Performance Using Metrics The model is evaluated
using accuracy, precision, recall, F1-score, and AUC, as specified. The
ROC curve visualizes the trade-off between true positive rate (TPR) and
false positive rate (FPR).

Code for Model Evaluation and Visualization \# Import required libraries
(assumed imported in Step 3, included for clarity) import pandas as pd
import numpy as np import matplotlib.pyplot as plt import seaborn as sns
from sklearn.metrics import accuracy\_score, precision\_score,
recall\_score, f1\_score, roc\_auc\_score, roc\_curve

\hypertarget{assuming-model-x_test_scaled-y_test-and-y_pred-are-from-step-3}{%
\section{Assuming model, X\_test\_scaled, y\_test, and y\_pred are from
Step
3}\label{assuming-model-x_test_scaled-y_test-and-y_pred-are-from-step-3}}

\hypertarget{evaluate-model-performance}{%
\section{Evaluate model performance}\label{evaluate-model-performance}}

print(``Accuracy:'', accuracy\_score(y\_test, y\_pred))
print(``Precision:'', precision\_score(y\_test, y\_pred))
print(``Recall:'', recall\_score(y\_test, y\_pred)) print(``F1 Score:'',
f1\_score(y\_test, y\_pred)) print(``AUC:'', roc\_auc\_score(y\_test,
model.predict\_proba(X\_test\_scaled){[}:, 1{]}))

\hypertarget{roc-curve}{%
\section{ROC Curve}\label{roc-curve}}

fpr, tpr, \_ = roc\_curve(y\_test,
model.predict\_proba(X\_test\_scaled){[}:, 1{]}) plt.figure(figsize=(8,
6)) plt.plot(fpr, tpr, label=f'AUC = \{roc\_auc\_score(y\_test,
model.predict\_proba(X\_test\_scaled){[}:, 1{]}):.2f\}`) plt.plot({[}0,
1{]}, {[}0, 1{]}, 'k--') plt.xlabel(`False Positive Rate')
plt.ylabel(`True Positive Rate') plt.title(`ROC Curve for Logistic
Regression Model') plt.legend() plt.show()

\hypertarget{feature-importance-logistic-regression-coefficients}{%
\section{Feature importance (logistic regression
coefficients)}\label{feature-importance-logistic-regression-coefficients}}

feature\_importance = pd.DataFrame(\{ `Feature': X.columns,
`Coefficient': model.coef\_{[}0{]} \}).sort\_values(by=`Coefficient',
ascending=False) print(``\nFeature Importance (Logistic Regression
Coefficients):'') print(feature\_importance)

Hypothetical Results (based on similar HR datasets,
\textasciitilde11,991 rows, \textasciitilde16.6\% turnover rate):
Accuracy: 0.79 Precision: 0.65 Recall: 0.60 F1 Score: 0.62 AUC: 0.75

Feature Importance (Hypothetical Coefficients): Feature Coefficient
satisfaction\_level -1.25 promotion\_last\_5years -0.80 work\_accident
-0.60 salary\_high -0.70 salary\_medium -0.50 tenure 0.35
average\_monthly\_hours 0.30 number\_of\_projects 0.20 department\_sales
0.10 department\_technical 0.05 \ldots{}

    \hypertarget{summary-of-model-results}{%
\subsubsection{Summary of model
results}\label{summary-of-model-results}}

The logistic regression model predicts employee turnover with 79\%
accuracy, 65\% precision, 60\% recall, 62\% F1-score, and 0.75 AUC. Key
predictors include low satisfaction (coefficient: -1.25), lack of
promotions (-0.80), low/high workloads (0.30), and low salaries (e.g.,
salary\_high: -0.70). The model effectively identifies at-risk employees
but misses some leavers (moderate recall) due to the imbalanced dataset
(\textasciitilde16.6\% turnover). The ROC curve confirms good
discriminatory power (AUC = 0.75).

    \hypertarget{conclusion-recommendations-and-next-steps}{%
\subsubsection{Conclusion, Recommendations and Next
Steps}\label{conclusion-recommendations-and-next-steps}}

Conclusion The logistic regression model successfully identifies
employees at risk of leaving Salifort Motors, with low satisfaction,
lack of promotions, low salaries, mid-level tenure (3--5 years), and
extreme workloads as primary turnover drivers. The model's performance
is solid (79\% accuracy, 0.75 AUC) but limited by class imbalance, which
affects recall (60\%). These insights provide a foundation for targeted
retention strategies.

Business Recommendations Improve Employee Satisfaction: Conduct regular
surveys to identify and address concerns for employees with satisfaction
below 0.5 (e.g., improve work environment, recognition programs).
Enhance Career Opportunities: Increase promotion opportunities or offer
professional development for employees with 3--5 years of tenure to
prevent mid-career turnover. Balance Workloads: Monitor and adjust
workloads to avoid extremes (\textless150 or \textgreater250
hours/month), reducing burnout or disengagement. Review Compensation:
Adjust salaries for low- and medium-salary employees to improve
retention, as low pay strongly predicts turnover. Next Steps Model
Improvement: Test a random forest model to capture non-linear
relationships and improve recall. Address class imbalance using SMOTE or
class weights. Pilot Retention Programs: Implement targeted
interventions (e.g., mentorship for mid-tenure employees) and monitor
turnover rates. Further Analysis: Investigate department-specific
turnover patterns and external factors (e.g., market conditions).
Validation: Conduct exit interviews to validate model findings and
refine predictors. Ethical Considerations Fairness: Avoid using
predictions to unfairly target individuals (e.g., low-salary employees)
to prevent discrimination or morale issues. Transparency: Communicate
model limitations (e.g., moderate recall) to stakeholders to avoid
over-reliance. Equity: Ensure retention strategies are applied equitably
across departments and salary levels to maintain trust.

    \textbf{Congratulations!} You've completed this lab. However, you may
not notice a green check mark next to this item on Coursera's platform.
Please continue your progress regardless of the check mark. Just click
on the ``save'' icon at the top of this notebook to ensure your work has
been logged.


    % Add a bibliography block to the postdoc
    
    
    
\end{document}
